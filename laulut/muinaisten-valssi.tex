\section{Muinaisten valssi}


Saimaan saaressa pikkuinen torppa,\\
uhraa portailla kultisti Miikkulainen,\\
mutta pintaan ei nouse kuin norppa,\\
on pettymys melkoinen.\\
Aallon alla se suunnisti kutsujan luo,\\
sille tuttua tutumpi on messu tuo.\\
Laulu kertoo näin mitä on yksin kun jää,\\
hyvin hylje sen ymmärtää.

Uhrata enää ei Cthulhulle päitä,\\
maailma modernisoitunut on.\\
Muinaisjäänteitä viiksekkäitä,\\
mies sekä hylje kumpikin on.

Saimaan saaressa pikkuinen torppa,\\
sinne muinaisten kulttinsa tahtonut ei.\\
Yksin kultisti jäi kuten norppa,\\
myös sen mestarin kohtalo vei.\\
Suuri Saimaa mut sata on hylkeitä vaan,\\
kohta jäljellä ei ehkä ainuttakaan.\\
Uhriveistänsä teroittaa Miikkulainen,\\
aatos mieleessään hirmuinen.

Uhrata enää ei Cthulhulle päitä,\\
maailma modernisoitunut on.\\
Muinaisjäänteitä viiksekkäitä,\\
mies sekä hylje kumpikin on.

Saimaan saaressa pikkuinen torppa,\\
karjuu loitsuja kultisti Miikkulainen.\\
Uhrikivellä hengetön norppa,\\
makaa hiljalleen kylmentyen.\\
Suuri Saimaa ja alta sen lainehien,\\
nousee kauhu niin vanha ja lonkeroinen.\\
Niin on kohtalo norpan kuin Miikkulaisen,\\
olla sukunsa viimeinen.

Viimeinkin uhrataan Cthulhulle päitä,\\
maailmaan palannut muinainen on.\\
Lounaaksi syötyjä pölkkypäitä\\
mies sekä hylje kumpikin on.